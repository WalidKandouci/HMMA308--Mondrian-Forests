\documentclass[a4paper,10pt]{article}
\usepackage[margin=2.5cm]{geometry}

\usepackage{amssymb,amsmath,amsthm}
\usepackage{color}
\usepackage{enumitem}
\usepackage{dsfont}
\usepackage{bm}

\usepackage[authoryear]{natbib}
\usepackage{float}
\usepackage[T1]{fontenc}
\usepackage[utf8]{inputenc}
\usepackage[french]{babel} 
\usepackage{amsmath}
\usepackage{amssymb}
%\usepackage[top=1.5cm, bottom=1.5cm, left=1.5cm, right=1.5cm]{geometry}
\usepackage{graphicx}
\usepackage{float}
\usepackage{multicol}
\usepackage{lipsum}
\usepackage{ragged2e}
\usepackage{eurosym}
\usepackage{indentfirst}
\usepackage{minted}
\usepackage{titlesec}
\usepackage{pifont}
\usepackage{url}
\usepackage{epsf}



\newtheorem{theorem}{Theorem}
\newtheorem{proposition}{Proposition}
\newtheorem{cor}{Corollary}
\theoremstyle{definition}
\newtheorem{definition}{Definition}
\newtheorem{remark}{Remark}
\newtheorem{example}{Example}
\newtheorem{claim}{Claim}
\newtheorem{lemma}{Lemma}


%%%%%%%%%%%%%%%%%%%%%%%%%%%%%%%%%%%%%%%%%%%%%%%%%%%%%%%%%%%%%%%%%%%%%%%%%%%%%%%
% Algorithms
%%%%%%%%%%%%%%%%%%%%%%%%%%%%%%%%%%%%%%%%%%%%%%%%%%%%%%%%%%%%%%%%%%%%%%%%%%%%%%%

\usepackage{algorithm}
\usepackage{algorithmic}
\usepackage[titlenumbered,ruled,noend,algo2e]{algorithm2e}
\newcommand\mycommfont[1]{\footnotesize\ttfamily\textcolor{blue}{#1}}
\SetCommentSty{mycommfont}
\SetEndCharOfAlgoLine{}


%%%%%%%%%%%%%%%%%%%%%%%%%%%%%%%%%%%%%%%%%%%%%%%%%%%%%%%%%%%%%%%%%%%%%%%%%%%%%%%
% Code
%%%%%%%%%%%%%%%%%%%%%%%%%%%%%%%%%%%%%%%%%%%%%%%%%%%%%%%%%%%%%%%%%%%%%%%%%%%%%%%


\usepackage{fancyvrb}                  % for fancy verbatim
\usepackage{textcomp}
\usepackage[space=true]{accsupp}
% requires the latest version of package accsupp
\newcommand{\copyablespace}{
    \BeginAccSupp{method=hex,unicode,ActualText=00A0}
\ %
    \EndAccSupp{}
}
\usepackage[procnames]{listings}
% \usepackage{setspace} % need for \setstretch{1}
\lstset{%
language   = python,%
 % basicstyle = \ttfamily\setstretch{1},%
basicstyle = \ttfamily,%
columns    = flexible,%
keywordstyle=\color{javared},
firstnumber=100,
frame=shadowbox,
showstringspaces=false,
morekeywords={import,from,class,def,for,while,if,is,in,elif,
else,not,and,or,print,break,continue,return,True,False,None,access,
as,del,except,exec,finally,global,import,lambda,pass,print,raise,try,assert,!=},
keywordstyle={\color{javared}\bfseries},
commentstyle=\color{javagreen}, %vsomb_col white comments
morecomment=[s][\color{javagreen}]{"""}{"""},
upquote=true,
%% style for number
numbers=none,
resetmargins=true,
xleftmargin=10pt,
linewidth= \linewidth,
numberstyle=\tiny,
stepnumber=1,
numbersep=8pt, %
frame=shadowbox,
rulesepcolor=\color{black},
procnamekeys={def,class},
procnamestyle=\color{blue}\textbf,
literate={á}{{\'a}}1
{à}{{\`a }}1
{ã}{{\~a}}1
{é}{{\'e}}1
{ê}{{\^e}}1
{è}{{\`e}}1
{í}{{\'i}}1
{î}{{\^i}}1
{ó}{{\'o}}1
{õ}{{\~o}}1
{ô}{{\^o}}1
{ú}{{\'u}}1
{ü}{{\"u}}1
{ç}{{\c{c}}}1
}

\definecolor{javared}{rgb}{0.6,0,0} % for strings
\definecolor{javagreen}{rgb}{0.25,0.5,0.35} % comments
\definecolor{javapurple}{rgb}{0.5,0,0.35} % keywords
\definecolor{javadocblue}{rgb}{0.25,0.35,0.75} % javadoc
\definecolor{marron}{rgb}{0.64,0.16,0.16}
\definecolor{orange_js}{RGB}{230,159,0}

\usepackage{times} % use Times

%\usepackage{../sty/shortcuts_js} % possibly adapted from https://github.com/josephsalmon/OrganizationFiles/sty/shortcuts_js.sty

%%%%%%%%%%%%%%%%%%%%%%%%%%%%%%%%%%%%%%%%%%%%%%%%%%%%%%%%%%%%%%%%%%%%%%%%%%%%%%%
% IMAGES
%%%%%%%%%%%%%%%%%%%%%%%%%%%%%%%%%%%%%%%%%%%%%%%%%%%%%%%%%%%%%%%%%%%%%%%%%%%%%%%

% Use prebuiltimages/ for images extracted from code (e.g. python)
% or to share images built from a software not available by the whole team (say matlab .fig, or inskcape .svg).
% .svg files should be stored in dir srcimages/ and built from moosetex if needed:
% https://www.charles-deledalle.fr/pages/moosetex.php
% NEVER (GIT) versions files in images/ : only prebuiltimages/ & srcimages/ !

\usepackage{graphicx} % For figures
\graphicspath{{images/}, {prebuiltimages/}}
\usepackage{subcaption}


%%%%%%%%%%%%%%%%%%%%%%%%%%%%%%%%%%%%%%%%%%%%%%%%%%%%%%%%%%%%%%%%%%%%%%%%%%%%%%%
% For citations
%%%%%%%%%%%%%%%%%%%%%%%%%%%%%%%%%%%%%%%%%%%%%%%%%%%%%%%%%%%%%%%%%%%%%%%%%%%%%%%


\usepackage{cleveref} % mandatory for no pbs with hyperlinks theorem etc...
\crefformat{equation}{Eq.~(#2#1#3)} % format for equations
\Crefformat{equation}{Equation~(#2#1#3)} % format for equations


%%%%%%%%%%%%%%%%%%%%%%%%%%%%%%%%%%%%%%%%%%%%%%%%%%%%%%%%%%%%%%%%%%%%%%%%%%%%%%%
% Header and document start
%%%%%%%%%%%%%%%%%%%%%%%%%%%%%%%%%%%%%%%%%%%%%%%%%%%%%%%%%%%%%%%%%%%%%%%%%%%%%%%


%\author{Walid Kandouci}
%\title{Comment fonctionne du modèle mixte linéaire
%}

\begin{document}

\begin{titlepage}
\newcommand{\HRule}{\rule{\linewidth}{0.5mm}}
\center
\textsc{\LARGE
Université de Montpellier
} \\[1cm]
\includegraphics[scale=0.4]{umontpellier_logo} \\[1cm]
\HRule \\[0.4cm]
{ \Huge \bfseries Les forêts aléatoires: Mondrian Forests \\[0.15cm] }
\HRule \\[0.4cm]
 \Large  HMMA 308: Apprentissage Statistique \\[13cm]
 

\LARGE Walid KANDOUCI

\end{titlepage}

%\maketitle

%\vskip 0.3in

%\begin{figure}[h] % h stands for here, ! forces even more...
%	\centering
%	\includegraphics[width=0.2\textwidth]{umontpellier_logo}
%	\caption{Illustration of a prebuiltimage available.}
%	\label{fig:umontpellier_logo}
%\end{figure}

%\begin{abstract}
%\input{content/abstract}
%\end{abstract}

\newpage

\tableofcontents

\newpage




%%%%%%%%%%%%%%%%%%%%%%%%%%%%%%%%%%%%%%%%%%%%%%%%%%%%%%%%%%%%%%%%%%%%%%%%%%%%%%%
% Sections in separated files
%%%%%%%%%%%%%%%%%%%%%%%%%%%%%%%%%%%%%%%%%%%%%%%%%%%%%%%%%%%%%%%%%%%%%%%%%%%%%%%

\newpage


\input{content/Chapters}
% ...more text here.
\section*{Code Python}
\subsection*{Mondrian Générateur}
\begin{lstlisting}[language=Python]
import numpy as np
import pandas as pd
import matplotlib.pyplot as plt
import matplotlib.path as mpath
import matplotlib.lines as mlines
import matplotlib.patches as mpatches
from matplotlib.collections import PatchCollection

%matplotlib inline

def dimensions(box, size=None):
    if box is None:
        if size is not None:
            return np.zeros(size)
        else: 
            raise ValueError('If box is not provided, you must specify size.')
            
    return np.diff(box, axis=1).flatten()

def linear_dimension(box):
    if box is None:
        return 0
    return dimensions(box).sum()

def interval_difference(outer_interval, inner_interval):
    lower_outer, upper_outer = outer_interval
    lower_inner, upper_inner = inner_interval
    return [lower_outer, lower_inner], [upper_inner, upper_outer]

def sample_interval_difference(outer_interval, inner_interval):
    intervals = interval_difference(outer_interval, inner_interval)
    dimensions = [np.diff(intervals[0])[0], np.diff(intervals[1])[0]]
    chosen_interval_index = np.random.choice(range(len(intervals)),
    p=dimensions/np.sum(dimensions))
    chosen_interval = intervals[chosen_interval_index]
    return np.random.uniform(low=chosen_interval[0], 
    high=chosen_interval[1], size=1)[0], chosen_interval_index

def random_axis(dimensions):
    return np.random.choice(range(len(dimensions)), 
                            p=dimensions/np.sum(dimensions))

def random_cut(box, axis):
    return np.random.uniform(low=self.box[axis][0], 
                             high=self.box[axis][1], 
                             size=1)[0]

def cost_next_cut(linear_dimension):
    return np.random.exponential(scale=1.0/linear_dimension, size=1)[0]

def new_cut_proposal(self):
    cost = self.cost_next_cut()
    axis = self.random_axis()
    cut_point = self.random_cut(axis)
    return cost, axis, cut_point

def cut_boxes(box, cut_axis, cut_point):
    left = box.copy()
    right = box.copy()
    low, high = box[cut_axis]

    if cut_point <= low or cut_point >= high:
        raise ValueError('Point is not in interval.')

    left[cut_axis] = [low, cut_point]
    right[cut_axis] = [cut_point, high]
    return left, right
    
class Mondrian(object):
    def __init__(self, box, budget):
        self.box = box
        self.budget = budget
        self.cut_point = None
        self.cut_axis = None
        self.cut_budget = None
        self.left = None
        self.right = None
        
    def extended_by(self, box):
        if self.box is None:
            return True
        return all((box[:, 0] <= self.box[:, 0]) & (box[:, 1] >= self.box[:, 1]))
    
    def contains(self, point):
        if self.box is None:
            return False
        return all(box[:,0] <= point) & (box[:,1] >= point)
    
    def has_cut(self):
        return self.cut_axis is not None
    
    def is_empty(self):
        return self.box is None
    
def grow_mondrian(box, budget, given_mondrian=None):
    if given_mondrian is None:
        given_mondrian = Mondrian(None, budget)

    if not given_mondrian.extended_by(box):
        raise ValueError('Incompatible boxes: given mondrian 
        box must be contained in new box.')
    
    mondrian = Mondrian(box, budget)
    
    cost = cost_next_cut(linear_dimension(box) -
    linear_dimension(given_mondrian.box))
    
    next_budget = budget - cost
    
    given_mondrian_next_budget = given_mondrian.cut_budget if
    given_mondrian.has_cut() else 0
    
    
    if next_budget < given_mondrian_next_budget:        
        if given_mondrian.has_cut():
            mondrian.cut_axis = given_mondrian.cut_axis
            mondrian.cut_point = given_mondrian.cut_point
            mondrian.cut_budget = given_mondrian_next_budget

            left, right = cut_boxes(box, mondrian.cut_axis,
            mondrian.cut_point)

            mondrian.left = grow_mondrian(left, 
            given_mondrian_next_budget, given_mondrian.left)
            mondrian.right = grow_mondrian(right, 
            given_mondrian_next_budget, given_mondrian.right)
    else:
        dimensions_outer = dimensions(box)
        dimensions_inner = dimensions(given_mondrian.box, size=len(box))
        mondrian.cut_axis = random_axis(dimensions_outer - dimensions_inner)
        outer_interval = box[mondrian.cut_axis]
        
        if given_mondrian.is_empty():
            inner_interval = [outer_interval[0], outer_interval[0]]
        else:
            inner_interval = given_mondrian.box[mondrian.cut_axis]

        mondrian.cut_point, cut_side = sample_interval_difference
        (outer_interval, inner_interval)
        mondrian.cut_budget = next_budget
        
        left, right = cut_boxes(box, mondrian.cut_axis, mondrian.cut_point)

        if cut_side: # entire given_mondrian to the left
            mondrian.left = grow_mondrian(left, next_budget, given_mondrian)
            mondrian.right = grow_mondrian(right, next_budget, 
            Mondrian(None, next_budget))
        else: # all given_mondrian to the right
            mondrian.left = grow_mondrian(left, next_budget, 
            Mondrian(None, next_budget))
            mondrian.right = grow_mondrian(right, 
            next_budget, given_mondrian)
            
    return mondrian
    
def get_random_color():
    return np.random.choice(['blue', 'red', 'yellow', 'white'], 1)[0]
    
def box_2d(box, color=None, alpha=None):
    low_x, high_x = box[0]
    low_y, high_y = box[1]
    width = high_x - low_x
    height = high_y - low_y
    
    if color is None:
        color = 'white'
    if alpha is None:
        alpha = 1
        
    lower_left_corner = np.array([low_x, low_y])
    
    return mpatches.Rectangle(lower_left_corner, width, height, 
                              color=color, ec='black', linewidth=2, 
                              alpha=alpha)

def boxes(m, box_collection, color=None):
    random_color = False
    
    if color == 'true_mondrian':
        random_color = True
        color = get_random_color()
    
    box_collection.append(box_2d(m.box, color))
    
    if m.has_cut():
        color = 'true_mondrian' if random_color else color
        boxes(m.left, box_collection, color)
        boxes(m.right, box_collection, color)
        
def plot_coloured_mondrian(m, ax, color=None, given_mondrian=None):
    if given_mondrian is None:
        given_mondrian = Mondrian(None, budget)
        
    box_collection = []
    boxes(m, box_collection, color)
    
    if not given_mondrian.is_empty():
        box_collection.append(box_2d(given_mondrian.box, 
                                     color='black', alpha=0.1))
        
    collection = PatchCollection(box_collection, match_original=True)
    ax.add_collection(collection)
    
    ax.axis('off')
    ax.autoscale()
    #plt.show()
    
def random_mondrians(box, budget, given_mondrian=None, 
color=None, rows=1, columns=1, figsize=(15, 15)):
    if given_mondrian is None:
        given_mondrian = Mondrian(None, budget)
        
    if rows == 1 and columns == 1:
        fig, ax = plt.subplots(figsize=figsize)  
        plot_coloured_mondrian(grow_mondrian(box, budget, given_mondrian), 
                               ax, color=color, given_mondrian=given_mondrian)
    else:
        fig, ax = plt.subplots(rows, columns, figsize=figsize)  

        for row in range(rows):
            for col in range(columns):
                if not given_mondrian.is_empty() and row == 0 and col ==0:
                    plot_coloured_mondrian(given_mondrian, 
                    ax[row, col], color=color)
                else:
                    plot_coloured_mondrian(grow_mondrian(box, budget,
                    given_mondrian), 
                                           ax[row, col], color=color,
                                           given_mondrian=given_mondrian)
                    
def growing_mondrians(initial_box, budget, 
rows=1, columns=1, figsize=(15, 15)):
    if rows == 1 and columns == 1:
        fig, ax = plt.subplots(figsize=figsize)  
        plot_coloured_mondrian(grow_mondrian(box, budget), 
                               ax, color=None)
    else:
        fig, ax = plt.subplots(rows, columns, figsize=figsize)  
        
        given_mondrian = Mondrian(None, budget)
        box = initial_box
        for row in range(rows):
            for col in range(columns):
                mondrian = grow_mondrian(box, budget, given_mondrian)
                plot_coloured_mondrian(mondrian, ax[row, col], color=None,
                given_mondrian=given_mondrian)
                given_mondrian = mondrian
                box = 2 * box    

random_mondrians(box, 1, rows=3, columns=3, color='true_mondrian')
plt.savefig("MondrianExmpl.pdf")
\end{lstlisting}
\subsection*{Mondrian Forest sur données "Boston Housing"}
\begin{lstlisting}[language=Python]
%matplotlib inline

import numpy as np
import pandas as pd
import statsmodels.api as sm
import statsmodels.formula.api as smf
import statsmodels.genmod
from statsmodels.tools.sm_exceptions import ConvergenceWarning
import matplotlib.pyplot as plt
from math import sqrt
import seaborn as sns
sns.set_palette("colorblind")
import sklearn.datasets 
from sklearn.model_selection import train_test_split
import skgarden
from skgarden import MondrianForestClassifier
from skgarden import MondrianForestRegressor
from skgarden import MondrianTreeClassifier
from skgarden import MondrianTreeRegressor
from sklearn.linear_model import LinearRegression
import sklearn

from sklearn.datasets import load_boston

boston = load_boston()

df_x=pd.DataFrame(boston.data, columns=boston.feature_names)
df_y=pd.DataFrame(boston.target)

df_x.head(5)

sns.distplot(df_x.AGE)
plt.savefig("Age.pdf") 

sns.distplot(df_x.RM)
plt.savefig("RM.pdf") 

sns.distplot(df_x.TAX)
plt.savefig("TAX.pdf") 

print(boston.DESCR)

# Mondrian Tree Regressor
mtr = MondrianForestRegressor()
mtr.fit(X, y)
y_mean, y_std = mtr.predict(X, return_std=True)
mtr.fit(X, y)
\end{lstlisting}



\end{document}
