\documentclass[unknownkeysallowed]{beamer}
\usepackage[french,english]{babel}
\input{Beamer_js}
\input{shortcuts_js}
%\usepackage{./OrganizationFiles/tex/sty/shortcuts_js}
\usepackage{csquotes}
\graphicspath{{./images/}}

\addbibresource{Bibliographie.bib}
\usepackage{enumerate}

\usepackage[T1]{fontenc}
\usepackage[utf8x]{inputenc}
%\usepackage{dsfont}
%\usepackage{bbold}
%\usepackage{stmaryrd}
%\languagepath{French}
%\usepackage{xcolor}
%\usepackage{pgf}
%\usepackage{tikz}



\usepackage[procnames]{listings}
% \usepackage{setspace} % need for \setstretch{1}
\lstset{%
language   = python,%
 % basicstyle = \ttfamily\setstretch{1},%
basicstyle = \ttfamily,%
columns    = flexible,%
keywordstyle=\color{javared},
firstnumber=100,
frame=shadowbox,
showstringspaces=false,
morekeywords={import,from,class,def,for,while,if,is,in,elif,
else,not,and,or,print,break,continue,return,True,False,None,access,
as,del,except,exec,finally,global,import,lambda,pass,print,raise,try,assert,!=},
keywordstyle={\color{javared}\bfseries},
commentstyle=\color{javagreen}, %vsomb_col white comments
morecomment=[s][\color{javagreen}]{"""}{"""},
upquote=true,
%% style for number
numbers=none,
resetmargins=true,
xleftmargin=10pt,
linewidth= \linewidth,
numberstyle=\tiny,
stepnumber=1,
numbersep=8pt, %
frame=shadowbox,
rulesepcolor=\color{black},
procnamekeys={def,class},
procnamestyle=\color{oneblue}\textbf,
literate={á}{{\'a}}1
{à}{{\`a }}1
{ã}{{\~a}}1
{é}{{\'e}}1
{ê}{{\^e}}1
{è}{{\`e}}1
{í}{{\'i}}1
{î}{{\^i}}1
{ó}{{\'o}}1
{õ}{{\~o}}1
{ô}{{\^o}}1
{ú}{{\'u}}1
{ü}{{\"u}}1
{ç}{{\c{c}}}1
}










\begin{document}


\begin{frame}[noframenumbering]
\thispagestyle{empty}
\bigskip
\bigskip
\begin{center}{
\LARGE\color{marron}
\textbf{Mondrian Forests}
\textbf{ }\\
\vspace{0.5cm}
}

\color{marron}
\textbf{HMMA 308 : Apprentissage Statistique}
\end{center}

\vspace{0.5cm}

\begin{center}
\textbf{KANDOUCI Walid} \\
\vspace{0.1cm}
\url{https://github.com/WalidKandouci/HMMA308--Mondrian-Forests}\\
\vspace{0.5cm}
Université de Montpellier \\
\end{center}

\centering
\includegraphics[width=0.13\textwidth]{Logo.pdf}
\end{frame}






%%%%%%%%%%%%%%%%%%%%%%%%%%%%%%%%%%%%%%%%%%%%%%%%%%%%%%%%%%%%%%%%%%%%%%%%%%%%%%%
%%%%%%%%%%%%%%%%%%%%%%             Headers               %%%%%%%%%%%%%%%%%%%%%%
%%%%%%%%%%%%%%%%%%%%%%%%%%%%%%%%%%%%%%%%%%%%%%%%%%%%%%%%%%%%%%%%%%%%%%%%%%%%%%%


% plan de la présentation
\begin{frame}
\frametitle{Sommaire}
\tableofcontents
\end{frame}


\section{Introduction}
\begin{frame}
\frametitle{Introduction}
\underline{\textbf{Rappel forêts aléatoires:}}
\begin{itemize}
    \item Algorithme bagging
    \item Bon résultat sur de vrai données
    \item simples à mettre en œuvre
\end{itemize}
\end{frame}

\begin{frame}
\frametitle{Introduction}
\underline{\textbf{Forêts Mondrian:}}
\begin{itemize}
    \item Nouvelle classe des forêts aléatoires
    \item Efficaces
    \item Offre meilleure estimation d'incertitude que les forêts aléatoires
\end{itemize}
\begin{figure}[H]
    \centering
    \includegraphics[width=10cm, height=1.5cm]{Images/Mondrian.jpg}
\end{figure}
\end{frame}


\section{Arbres de Mondiran}
\begin{frame}
\frametitle{Arbres de Mondrian}
\underline{\textbf{L'approche:}}
\begin{itemize}
    \item chaque noeud $j$ a exactement un noeud parent, sauf pour un noeud racine distingué $\epsilon$ qui n’a pas de parents
    \item chaque noeud $j$ est le parent d’exactement zéro ou deux noeuds enfants : (le noeud de gauche "$left(j)$" et le noeud de droite "$right(j)$")
\end{itemize}
\end{frame}

\begin{frame}
\frametitle{Arbres de Mondrian}

\begin{itemize}
    \item $(T, \delta, \xi)$ un arbre de décision
    \item  parent $(j):$ le parent du noeud $j$
    \item $N(j): 1$ l'indice de nos données d'apprentissage au point $j(N(j)=\{n \in\{1, \ldots, N\}: x_{n} \in B_{j}\})$
    \item  $\mathcal{D}_{N(j)}=\left\{\boldsymbol{X}_{N(j)}, Y_{N(j)}\right\}:$ les caractéristiques et les étiquettes des points de données d'apprentissage au noeud $j$
    \item $\cdot \ell_{j d}^{x}$ et $u_{j d}^{x}:$ les bornes inférieurs et supérieurs de nos donnes d'apprentissage au noeud $j$ le long de la dimension $d$
    \item $\cdot B_{j}^{x}=\left(\ell_{j 1}^{x}, u_{j 1}^{x}\right] \times \ldots \times\left(\ell_{j D}^{x}, u_{j D}^{x}\right] \subseteq B_{j}:$ le plus petit rectangle qui entoure les données d'apprentissage
au noeud $j$
    
\end{itemize}
\end{frame}

\begin{frame}
\frametitle{Arbres de Mondrian}
\underline{\textbf{L'algorithme:}}

\begin{figure}[H]
    \centering
    \includegraphics[width=10cm, height=5cm]{Images/ALGO1ET2.PNG}
\end{figure}

\end{frame}


\begin{frame}
\frametitle{Arbres de Mondrian}
Les arbres de Mondrian diffèrent des arbres de décision comme suit:

\begin{itemize}
    \item Les divisions sont échantillonnées indépendamment des $Y_{N(j)}$
    \item Chaque noeud $j$ est associé avec un temps de division $\tau_{j}$
    \item $\lambda$ contrôle le nombre totale des divisions
    \item la division représenté par un noeud interne $j$ ne tient que dans $B_{j}^{x}$ et non pas $B_{j}$
\end{itemize}
\end{frame}

\section{Application}
\begin{frame}[fragile]
\frametitle{Application}
\underline{\textbf{Fonction "$random\_mondrian$"}}

\begin{figure}[H]
  \centering
  \begin{subfigure}{.3\linewidth}
    \centering
    \includegraphics[width = \linewidth]{Images/Mondrianbudget2.pdf}
  \end{subfigure}%
  \hspace{1em}% Space between image A and B
  \begin{subfigure}{.3\linewidth}
    \centering
    \includegraphics[width = \linewidth]{Images/Mondrianbudget10.pdf}
  \end{subfigure}%
  \hspace{2em}% Space between image B and C
  \begin{subfigure}{.3\linewidth}
    \centering
    \includegraphics[width = \linewidth]{Images/Mondrianbudget50.pdf}
  \end{subfigure}
  \caption{Exemple générateur aléatoire Mondrian (budget=2,10,50)}
\end{figure}
\end{frame}

\section{Conclusion}
\begin{frame}
\frametitle{Conclusion}
\begin{itemize}
    \item Forêts de Mondrian = Processus de Mondrian + Forêts aléatoires
    \item Peut fonctionner en mode batch ou en mode en ligne
    \item Meilleur estimation d'incertitude que les forêts aléatoires
\end{itemize}
\end{frame}

\end{document}

